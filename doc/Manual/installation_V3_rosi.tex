\chapter*{Installation}
This installation guidelines are for the new version of DuMu$^{x}$, version 3, coupled with CRootBox, in Linux systems (e.g. Ubuntu). 

\section*{Required compilers and tools}
%
If on a recent Ubuntu system, the c++ compiler and python that come with the distribution are recent enough. Otherwise, please make sure you have a recent c++ compiler (e.g. \lstinline{sudo apt-get install clang}), fortran compiler (sudo apt-get install gfortran) and python3 (e.g. \lstinline{sudo apt-get install python3.7}). 

- Install git: \\
\lstinline{sudo apt-get install git}\\
- Install cmake:\\
\lstinline{sudo apt-get install cmake}\\
- Install libboost:\\
\lstinline{sudo apt-get install libboost-all-dev}\\
- Install pip:\\
\lstinline{sudo apt-get install python3-pip}\\
- Install the python package numpy:\\
\lstinline{pip3 install numpy}\\
- Install the python package scipy:\\
\lstinline{pip3 install scipy}\\
- Install the python package matplotlib:\\
\lstinline{pip3 install matplotlib}\footnote{Known bug in ubuntu 18.04: needs sudo apt-get install libfreetype6-dev libxft-dev installed before.}\\
- Install the python package VTK:\\alug
\lstinline {pip3 install vtk}\\
- Install the java runtime environment:\\
\lstinline{sudo apt-get install default-jre}\\
- Install Paraview\\
\lstinline{sudo apt-get install paraview}\\


\lstinline{}\\

\section*{DuMu$^x$ installation}
In all dune modules we stay in version 2.6, the latest stable release version. 

- Create a DuMu$^x$ working folder\\
\lstinline{mkdir DUMUX}\\
\lstinline{cd DUMUX}\\
- Download DUNE core modules:\\
\texttt{git clone https://gitlab.dune-project.org/core/dune-common.git}\\
    \hspace{\parindent} \texttt{cd dune-common}\\
    \hspace{\parindent} \texttt{git checkout releases/2.6}\\
		\hspace{\parindent} \texttt{cd ..}\\
\texttt{git clone https://gitlab.dune-project.org/core/dune-geometry.git}\\
    \hspace{\parindent} \texttt{cd dune-geometry}\\
    \hspace{\parindent} \texttt{git checkout releases/2.6}\\
		\hspace{\parindent} \texttt{cd ..}\\
\texttt{git clone https://gitlab.dune-project.org/core/dune-grid.git}\\
    \hspace{\parindent} \texttt{cd dune-grid}\\
    \hspace{\parindent} \texttt{git checkout releases/2.6}\\
		\hspace{\parindent} \texttt{cd ..}\\
\texttt{git clone https://gitlab.dune-project.org/core/dune-istl.git}\\
    \hspace{\parindent} \texttt{cd dune-istl}\\
    \hspace{\parindent} \texttt{git checkout releases/2.6}\\
		\hspace{\parindent} \texttt{cd ..}\\
\texttt{git clone https://gitlab.dune-project.org/core/dune-localfunctions.git}\\
    \hspace{\parindent} \texttt{cd dune-localfunctions}\\
    \hspace{\parindent} \texttt{git checkout releases/2.6}\\
		\hspace{\parindent} \texttt{cd ..}\\
- Download DUNE external modules:\\
\texttt{git clone https://gitlab.dune-project.org/extensions/dune-foamgrid.git}\\
    \hspace{\parindent} \texttt{cd dune-foamgrid}\\
    \hspace{\parindent} \texttt{git checkout releases/2.6}\\
		\hspace{\parindent} \texttt{cd ..}\\
\texttt{git clone https://gitlab.dune-project.org/extensions/dune-grid-glue.git}\\
    \hspace{\parindent} \texttt{cd dune-grid-glue}\\
    \hspace{\parindent} \texttt{git checkout master\footnote{There is no release 2.6}}\\
		\hspace{\parindent} \texttt{cd ..}\\

-Download dumux and dumux-rosi and the grid manageres alugrid, uggrid and spgrid:\\
\texttt{git clone https://git.iws.uni-stuttgart.de/dumux-repositories/dumux.git}\\
    \hspace{\parindent} \texttt{cd dumux}\\
    \hspace{\parindent} \texttt{git checkout releases/3.0}\\
		\hspace{\parindent} \texttt{cd ..}\\
\texttt{git clone https://github.com/Plant-Root-Soil-Interactions-Modelling/dumux-rosi.git}\\
    \hspace{\parindent} \texttt{cd dumux-rosi}\\
    \hspace{\parindent} \texttt{git checkout master}\\
		\hspace{\parindent} \texttt{cd ..}\\
\texttt{git clone https://gitlab.dune-project.org/extensions/dune-alugrid.git}\\
    \hspace{\parindent} \texttt{cd dune-alugrid}\\
    \hspace{\parindent} \texttt{git checkout releases/2.6}\\
		\hspace{\parindent} \texttt{cd ..}\\
\texttt{git clone https://gitlab.dune-project.org/staging/dune-uggrid}\\
    \hspace{\parindent} \texttt{cd dune-uggrid}\\
    \hspace{\parindent} \texttt{git checkout releases/2.6}\\
		\hspace{\parindent} \texttt{cd ..}\\
\texttt{git clone https://gitlab.dune-project.org/extensions/dune-spgrid}\\
    \hspace{\parindent} \texttt{cd dune-spgrid}\\
    \hspace{\parindent} \texttt{git checkout releases/2.6}\\
		\hspace{\parindent} \texttt{cd ..}\\				

-Download CRootBox (only needed if root growth is used):\\
\texttt{git clone https://github.com/Plant-Root-Soil-Interactions-Modelling/CRootBox.git}\\
    \hspace{\parindent} \texttt{cd CRootBox}\\
    \hspace{\parindent} \texttt{git checkout master}\\
		\hspace{\parindent} \texttt{cd ..}\\
To build CRootBox and its python shared library, move again into the CRootBox folder and type into the console:\\
\lstinline {cd CRootBox}\\
\lstinline{cmake .}\footnote{It may be necessary on your installation to check the CRootBox/src/CMakeLists.txt file regarding required python version and out-commenting line 34.}\\
\lstinline{make}\\
(If building CRootBox on the cluster, two lines in the file \lstinline{CRootBox/CMakeLists.txt} need to be outcommented before:\\ 
\lstinline{set(CMAKE_C_COMPILER "/usr/bin/gcc")}\\
\lstinline{set(CMAKE_CXX_COMPILER "/usr/bin/g++")})\\

Now build DuMu$^{x}$ with the CRootBox module: \\
-The configuration file \lstinline{cmake.opts} is stored in the dumux folder. Move a copy of this file to your DuMu$^x$ working folder (one level up, DUMUX)\\

- To build all downloaded modules and check whether all dependencies and prerequisites are met, run dunecontrol:\\
\lstinline{cd ..} \\
\lstinline{./dune-common/bin/dunecontrol --opts=cmake.opts all}\\

Installation done! Good luck!

\section*{Running an example}

\begin{lstlisting}
cd  dumux-rosi/build-cmake/rosi_benchmarking/soil
make richards1d      # outcomment if executable is already available
./richards1d benchmarks_1d/b1a.input   # run executable with specific input parameter file
\end{lstlisting}

\section*{Installing and running an example on the agrocluster}
- Before installing or running DuMu$^x$ on the agrocluster, it is required to type the command \lstinline{module load dumux} into the console. This sets the compiler versions and other tools to more recent versions than the standard versions of the agrocluster.\\
- On the cluster, another onfiguration file \lstinline{optim_cluster.opts} is used. Copy this file to the file to your DuMu$^x$ working folder (one level up).\\ 
- To build or run an example on the agrocluster, create a pbs file in your working folder that will put your job in the cluster queue

For example \lstinline{queue_my_job.pbs}

\begin{lstlisting}
#!/bin/sh
#
#These commands set up the Grid Environment for your job:
#PBS -N DUMUX 
#PBS -l nodes=1:ppn=1,walltime=200:00:00,pvmem=200gb
#PBS -q batch\\
#PBS -M a.schnepf@fz-juelich.de
#PBS -m abe

module load dumux 
cd  \$HOME/DUMUX/dumux-rosi/build-cmake/rosi_benchmarking/soil
make richards      
./richards benchmarks_1d/b1a.input     
\end{lstlisting}


To start the job, run this file in your working folder with the command \\
\lstinline{qsub queue_my_job.pbs}

Use Filezilla to move the results to your local machine and use Paraview to visualize them. 

If you need to install additional python packages (e.g. scipy) on the cluster (without root access), you may do so by using the \lstinline{--user} command: \\
\lstinline{pip3 install --user scipy}
