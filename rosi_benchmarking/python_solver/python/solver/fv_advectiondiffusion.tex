\documentclass[a4paper,10pt]{article}
\usepackage[utf8]{inputenc}

%opening
\title{Discretisation of the advection diffusion equation}
\author{}

\begin{document}

\maketitle

\section{Equation}

The advection diffusion equation linear buffering is given by
\begin{eqnarray}
 (b+\theta) \frac{\partial c}{\partial t}  = \mathbf{\nabla} \cdot (D_e \mathbf{\nabla} c) - \mathbf{\nabla} \cdot (\mathbf{u} c) + R,  
\end{eqnarray}
where $t$ is the time [day], $c$  solute concentration [g/cm$^3$], $b$ buffer power [1], $\theta$ water content [1], $D_{eff}$ effective diffusion [cm$^2$], $\mathbf{u}$ velocity field [cm/day], and $R$ reaction term [g/cm$^3$/day]    

The effective diffusion $D_e$ is often given as function of the water content, e.g.
\begin{eqnarray}
 D_e = D \theta  \tau,  
\end{eqnarray}
where D is the diffusion constant of the solute in water, $\theta$ is the variable water conten (e.g. from the Richards equation) and $\tau$ is a constant that describes ???

The term $R$ could be a source or sink from a chemical reaction with another solute. For now we will ignore $R$.  

\section{Finite volume discretisation in space}

For a single control volume which will be a single cell with index $i$ in the finite volume scheme, the divergence can written as sum of volume fluxes entering or leaving the volume
\begin{eqnarray}
 (b+\theta_i) \frac{\partial c_i}{\partial t}  = \sum_{j \in N(i)} \frac{a_{ij}}{v_i}  F_{ij} - \frac{a_{ij}}{v_i}(\mathbf{u}_{ij} \cdot \mathbf{n}_{ij}) c_i + R_i,  \label{eqn:fv}
\end{eqnarray}
where $N(i)$ are the indices of the neigbouring cells of $i$, $a_{ij}$ is the area of the face between the cells $i$ and $j$ [cm$^2$], $v_i$ volume of cell $i$ [cm$^3$], $F_{ij}$ the mass flux between the cells $i$ and $j$ [g/cm$^2$/day], $\mathbf{u}_{ij}$ is the velocity at the center of the face, and $\mathbf{n}_{ij}$ is the outward normal ??? [1].

The value $\frac{a_{ij}}{v_i}$ can be computed exactly from the grid geometry. 

We approximate the flux $F_{ij}$ by first order finite difference
\begin{eqnarray}
F_{ij}  \approx \frac12 (D_{e,i} + D_{e,j})\frac{c_i - c_j} {d_{ij}},   
\end{eqnarray}
where $d_{ij}$ is the distance between the two cell centers with index $i$ and $j$ [cm]. Note that for equidisant grid this equals a ???. 

The velocity is approximated
\begin{eqnarray}
\mathbf{u}_{ij} \approx \frac12 (\mathbf{u}_i + \mathbf{u}_j).  
\end{eqnarray}

For mass conservation it must be assured that $a_{ij} = a_{ji}$,  $F_{ij} = -F_{ji}$,  $u_{ij}= u_{ij}$

Inserting the approximations in Eqn (\ref{eqn:fv}) yields
\begin{eqnarray}
 (b+\theta_i) \frac{\partial c_i}{\partial t}  =\sum_{j \in N(i)}  \frac{a_{ij}}{v_i} \left( \frac12 (D_{e,i} + D_{e,j})\frac{c_i - c_j} {d_{ij}},  - (\frac12 (\mathbf{u}_i + \mathbf{u}_j).\cdot \mathbf{n}_{ij}) c_i \right) + R_i,  \label{eqn:fv2}
\end{eqnarray}

This equation can be written as matrix vector multiplication
\begin{eqnarray}
\frac{\partial \mathbf{c}}{\partial t}  = A \mathbf{c},  \label{eqn:matrix}, 
\end{eqnarray}
$A=(a_{ij})_{ij}$ with diagonal entries
\begin{eqnarray}
a_{ii} = 
\end{eqnarray}
and secondary diagonals
\begin{eqnarray}
a_{ij} = 
\end{eqnarray}

Such finite volumes schemes work for any grid dimensions and for cylindric or branched geometries. Information needed is neigbourhood information (), area per volume (), distance between cell centers ().

If the grid is not equidistant, the appoximations of $D_e$ and $\mathbf{u}$ will introduce some error. 


\section{Discretisation in time}

\section{Implementation notes}


\end{document}
